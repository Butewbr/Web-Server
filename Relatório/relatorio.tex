\documentclass[a4paper,12pt]{article}

% template by Bernardo Pandolfi Costa

\usepackage{graphicx}
\usepackage[
top=3cm,
bottom=2cm,
left=3cm,
right=2cm,
]{geometry}
\usepackage{pdfpages}
\usepackage{hyperref}
\usepackage{indentfirst}
\usepackage{listings}
\usepackage{xcolor}
\usepackage{textcomp}
\usepackage[portuguese]{babel}
\usepackage[section]{placeins}
\usepackage[utf8]{inputenc}
\usepackage[T1]{fontenc}
\usepackage{fancyhdr}
\usepackage{lipsum}
% Turn on the style
\pagestyle{fancy}
\usepackage{float}
\usepackage{amsmath}
\usepackage{steinmetz}
\fancyfoot{}

\fancyfoot[R]{\thepage}
\setlength{\headheight}{16pt}% ...at least 51.60004pt

\definecolor{codegreen}{rgb}{0.48,0.7,0.42}
\definecolor{codegray}{rgb}{0.5,0.5,0.5}
\definecolor{codepurple}{rgb}{0.5,0,0.82}
\definecolor{codeblue}{rgb}{0.5, 0.64, 0.74}
\definecolor{backcolour}{rgb}{0.95,0.95,0.95}

\lstdefinestyle{codestyle}{
    backgroundcolor=\color{backcolour},   
    commentstyle=\color{codegray},
    keywordstyle=\color{codepurple},
    numberstyle=\tiny\color{codeblue},
    stringstyle=\color{codegreen},
   	breakatwhitespace=false,         
    breaklines=true,                 
    captionpos=b,                    
    keepspaces=true,                 
    numbers=left,                    
    numbersep=5pt,                  
    showspaces=false,                
    showstringspaces=false,
    showtabs=false,                  
    tabsize=2,
    basicstyle=\ttfamily
}
\usepackage[utf8]{inputenc}
\usepackage[T1]{fontenc}

\lstset{
inputencoding=utf8,
upquote=true,
style=codestyle,
    literate=%
    {á}{{\'a}}1
    {à}{{\`a}}1
    {ã}{{\~a}}1
    {â}{{\^a}}1
    {é}{{\'e}}1
    {ê}{{\^e}}1
    {í}{{\'i}}1
    {ó}{{\'o}}1
    {õ}{{\~o}}1
    {ô}{{\^o}}1
    {ú}{{\'u}}1
    {ü}{{\"u}}1
    {ç}{{\c{c}}}1
    {Á}{{\'A}}1
    {À}{{\`A}}1
    {Ã}{{\~A}}1
    {Â}{{\^A}}1
    {É}{{\'E}}1
    {Ê}{{\^E}}1
    {Í}{{\'I}}1
    {Ó}{{\'O}}1
    {Õ}{{\~O}}1
    {Ô}{{\^O}}1
    {Ú}{{\'U}}1
    {Ü}{{\"U}}1
    {Ç}{{\c{C}}}1
}

\usepackage{helvet}
\renewcommand{\familydefault}{\sfdefault}

\hypersetup{
    colorlinks,
    citecolor=black,
    filecolor=black,
    linkcolor=black,
    urlcolor=blue
}

\author{Bernardo Pandolfi Costa}
\title{Servidor Web}

\renewcommand{\contentsname}{Sumário}
\renewcommand{\figurename}{Figura}


\begin{document}
\makeatletter
\begin{titlepage}
    \begin{center}
    \includegraphics[width=64px]{./assets/vertical_sigla_fundo_claro.png}\\
        \small
        \vspace{0.25cm}
        \textbf{UNIVERSIDADE FEDERAL DE SANTA CATARINA}\\
		\textbf{CENTRO DE CIÊNCIAS, TECNOLOGIAS E SAÚDE}\\
		\textbf{DEPARTAMENTO DE COMPUTAÇÃO}\\
		\textbf{CURSO DE ENGENHARIA DE COMPUTAÇÃO}\\
        \vspace{7cm}
        \Large
        \textbf{\@title}\\
        \vspace{0.5cm}
        \normalsize
        \textbf{\@author}\\
        \vspace{1.5cm}
        \small
        Disciplina: Sistemas Operacionais\\
        Professor: Roberto Rodriges Filho
        \vfill
        Araranguá\\
        \today
    \end{center}
\end{titlepage}
\makeatother

\tableofcontents
\newpage

\section{Exercício 1}
Para este problema, foram feitas alterações aos códigos exemplos no arquivo da lista, sendo a alteração principal foi a adição de um loop, tanto para o cliente quanto para o servidor.
\subsection{Servidor}
No programa do servidor, o socket continuará ouvindo a porta 2023 até que o cliente envie uma mensagem.
Ao receber uma mensagem, o código a imprimirá no terminal e uma resposta é enviada ao cliente e o programa retorna ao estado de ouvindo.
Existe um comando que o cliente pode enviar que é o 'Stop'. O comando 'Stop' é interpretado pelo servidor para quebrar o loop e desligar o servidor.
\subsection{Cliente}
Para o programa do cliente, um loop similar foi implementado. Neste, o usuário será confrontado por um input para digitar a mensagem a ser enviada ao servidor.
O programa também imprime a resposta enviada pelo servidor e possui um passo de verificação para analisar se o servidor foi desligado ou não.
Esta verificação é feita da seguinte forma: se nenhuma resposta chegou, assume-se que o servidor não esteja no ar. 
Se a mensagem retornada é a resposta ao 'Stop', isto é, um aviso de desligamento do servidor, consideramos que o servidor será desligado. Quando esta verificação é verdadeira, o programa do cliente finaliza.
\section{Exercício 2}
A resolução deste exercício foi feita a partir de uma adaptação do código do servidor do exercício 1. O loop continua implementado, porém neste caso a resposta será sempre "HTTP/1.0 200 OK \textbackslash n\textbackslash n<h1>Hello World</h1>". Como o cliente neste caso é o browser, não foi implementado um programa para o cliente.
\section{Exercício 3}
Para o exercício 3, dois arquivos novos tiveram de ser criados: 404.html e index.html. Esses arquivos são usados para enviar a página html como resposta ao cliente com base na rota especificada na requisição. A documentação linha por linha do código está em forma de comentários no próprio código.
\subsection{Servidor}
O servidor implementado para este problema envolve quatro funções. Uma para   cada rota e uma para lidar com a requisição do cliente. Existem três rotas no servidor, a home (acessada pelo / ou por /index.html), a /request que apenas apresenta a requisição do cliente no navegador e a 404, para qualquer rota inexistente. A função que lida com a requisição é a função colocada em cada thread. Sempre que uma requisição chega ao servidor, uma nova thread é criada a partir desta função. Essa função checa a rota requisitada pelo cliente e retorna uma resposta chamando a respectiva função da rota.
\subsection{Cliente}
No momento que o código do cliente é rodado, o usuário será pedido a quantidade de requisições que serão feitas ao mesmo tempo para o servidor. Cada requisição, automaticamente, pedirá por diferentes rotas para testar cada rota implementada.
\end{document}